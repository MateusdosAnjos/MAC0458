\documentclass[12pt]{article}

\usepackage[utf8]{inputenc}
\usepackage[portuguese]{babel}

\title{Relatório 6: Marco Civil}
\author{Mateus Agostinho dos Anjos\\NUSP:9298191}
\date{\today}

\begin{document}
	\maketitle
		\paragraph{}
			A palestra cujo tema foi Marco Civil teve seu início com o palestrante
			Rafael Zanatta introduzindo brevemente o conceito de Marco Civil, 
			definindo-o como a lei que regula o uso da internet no Brasil por meio da
			previsão de princípios, garantias , direitos e deveres para quem usa a rede,
			bem como da determinação de diretrizes para a atuação do Estado. Após 
			esse início tivemos uma contextualização sobre o momento em que se
			começou a formar o conceito de Marco Civil com o palestrante discorrendo
			sobre a lei Azeredo, apelidada  de AI-5 digital, e sobre leis referentes ao
			cibercrime.
		\paragraph{}
			Após essa introdução vimos o que o Marco Civil \textbf{não} é:
			\begin{itemize}
				\item[•]
					Limitação de expressão
				\item[•]
					Lei de proteção a dados pessoais
				\item[•]
					Lei sobre direitos autorais	
				\item[•]
					Uma pessoa	
			\end{itemize}	
		\paragraph{}
			A fim de exemplificar a atuação do Marco Civil foi lembrado
			o "caso Cicarelli". A exposição de imagens da famosa em um vídeo no 
			YouTube resultou em um processo contra a empresa e um pedido de
			retirada dos vídeos que circulavam pela rede, porém como o YouTube
			não tem controle sobre o conteúdo dos vídeos e serve apenas como uma
			plataforma de transmissão o Marco Civil ajudou na defesa da empresa, 
			transferindo a responsabilidade dos vídeos para quem fez o upload e não
			para a empresa que os exibiu. Após o ocorrido a lei ganhou força e apoio
			de empresas, pois serve como proteção para novas ideias, uma vez que
			o responsável pelo mal uso de uma plataforma interessante e bacana nem
			sempre será quem criou a plataforma. 
		\paragraph{}
			Apesar deste apoio a aceitação da nova lei passou por dificuldades nos 
			momentos finais devido a grande divisão de ideias e conflitos políticos
			em 2014. Acreditava-se que o Marco Civil iria restringir o acesso à
			informação ou até servir como forma de censura ou manipulação de massas
			por meio da Internet. Entretanto a lei fora aprovada e se mostrou
			importante na manutenção da neutralidade da rede e evitando que
			os serviços de internet se assemelhassem a provedores de TV a cabo em que
			o acesso a informações é limitado por planos, ou seja, pelo poder econômico
			dos usuários.
		\paragraph{}
			Neste momento  Rafael Zanatta discorreu um pouco sobre o caso
			Snowden e a revelação da espionagem americana, dando muita força
			para o Marco Civil como forma de regularizar o uso da rede. Neste
			momento tivemos uma discussão bem interessante sobre os limites
			que a Internet tem ou deveria ter. Espionar dados de países faz com que
			determinadas nações ganhem muita vantagem política em relação a outras
			e consigam se	 manter no poder absoluto. Além da espionagem de países
			a coleta de dados pessoais também é um fator interessante a ser 
			questionado, até que ponto grandes empresas podem ou devem coletar
			nossas informações? quais informações pessoais deveriam estar na rede
			para serem acessadas? Como tais questões fogem um pouco do escopo
			de Marco Civil, não iremos abordá-las a fundo neste texto.
		\paragraph{}
			A palestra foi finalizada com a consolidação do Marco Civil a partir do
			momento que a CGI (Comitê Gestor da Internet) influenciou nos contornos
			da lei e o apoio importante de Gilberto Gil foi dado via Cultura Digital, 
			além do apoio de entidades civis como famosos e ONG's.	
		\paragraph{}
			Portanto, essa palestra nos mostrou um pouco mais do impacto da Internet
			em nossos diferentes ambientes (políticos, sociais, econômicos) enquanto
			alguns políticos acusavam Marco Civil de censura, outros utilizavam a rede
			para espionagem e obtenção de vantagens. A rede foi utilizada como forma
			de fofoca ao publicar vídeos de famosos	 e é uma enorme fonte de renda
			se levarmos em consideração as grandes empresas que trabalham com 
			plataformas online. Tudo isso deve ser entendido e controlado, para garantir
			que haja harmonia na utilização da rede independente se seu uso for
			recreativo ou com viés econômico, devemos garantir o acesso livre à todas
			as informações e que a internet não seja utilizada de forma que cause
			danos de qualquer tipo a seus usuários. 	
\end{document}