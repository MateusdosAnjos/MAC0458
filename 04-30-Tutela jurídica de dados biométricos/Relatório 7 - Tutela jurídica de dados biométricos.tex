\documentclass[12pt]{article}

\usepackage[utf8]{inputenc}
\usepackage[portuguese]{babel}

\title{Relatório 7: Tutela jurídica de dados biométricos no Brasil}
\author{Mateus Agostinho dos Anjos\\NUSP: 9298191}
\date{\today}

\begin{document}
	\maketitle
	\paragraph{}
		Nesta palestra sobre Tutela jurídica de dados biométricos no Brasil
		o palestrante, Matheus Treuk, norteou nossa discussão começando com uma
		definição etimológica da palavra \textit{Biometria}, que significa, de maneira 
		simplificada, "medida humana".  Definido isso, categorizou biometria em
		\textit{fisiológicas} e \textit{fisionômicas}.
	\paragraph{}
		Dando sequência à introdução vimos que o conceito de biometria pode ser
		aplicado desde a pré história, pois os seres reconheciam uns aos outros pelas
		características faciais e continuando os relatos históricos vimos que na 
		mesopotâmia antiga já havia início do reconhecimento de digitais. Já a partir de
		1971 a geometria da mão passou a ser utilizada, em 1994 a íris humana e
		em 2010 o \textit{Facebook} começou com o reconhecimento facial no
		aplicativo de marcar pessoas, introduzindo uma questão sobre dados 
		pessoais e captação de informações biométricas sem que a maioria das
		pessoas saiba disso.
	\paragraph{}
		Após a introdução histórica, Matheus Treuk trouxe a discussão para o contexto
		brasileiro em que não existe uma definição de dados biométricos apresentando
		também alguns conceitos chaves sobre biometria. A partir deste momento
		tivemos um breve exemplo de como uma possível captura de dados biométricos
		é feita: a primeira medição serve como cadastro e salva os dados iniciais, após
		isso é feita uma segunda medição que servirá como comparação, para que
		sejam identificadas as características biométricas mais importantes e assim
		os dados são salvos de forma otimizada.
	\paragraph{}
		Devemos lembrar que não basta fazer duas medições de quaisquer dados do
		corpo humano, para que um elemento seja um dado biométrico, portanto capaz
		de identificar um ser humano, ele deve possuir certas características, são
		elas:
		\begin{itemize}
			\item[•]
				Universal - Todas ou quase todas as pessoas possuem
			\item[•]
				Única - Devem ser apresentadas de forma única ou quase única em cada
				pessoa
			\item[•]
				Permanente - Permanente ou muito difícil de mudar		
		\end{itemize}			
	\paragraph{}
		Além destas características relacionadas aos dados biométricos vimos que 
		existem duas funções biométricas:
		\begin{itemize}
			\item[•]
				Identificação - de 1 para N
			\item[•]
				Verificação - de 1 para 1	
		\end{itemize}
	\paragraph{}
		Depois das descrições feitas sobre dados biométricos o palestrante levantou
		algumas questões para pensarmos a respeito das consequências do uso
		de dados biométricos em nossa vida social:
		\begin{itemize}
			\item[•]
				Discriminação - por exemplo: Como fica o cadastro de digitais de
				pessoas que não possuem mãos?
			\item[•]
				Controle Social - o fim do anonimato torna-se cada vez mais próximo
				uma vez que empresas terão dados biométricos capazes de identificar
				univocamente um indivíduo.	 
			\item[•]
				Controle Político - A tecnologia de reconhecimento facial já foi 
				utilizada no Irã para identificar manifestantes.	
		\end{itemize}
	\paragraph{}
		Partindo para os momentos finais da palestra entramos no panorama regulatório
		sobre os dados biométricos. O direito público sobre esses dados estão
		relacionados ao Estado, ou seja, o Estado se responsabiliza em não
		disponibilizar seus dados para terceiros e este direito é garantido
		por lei: Lei Geral de Proteção de Dados Pessoais.
		No ambito privado não existem leis que protegem os dados biométricos
		pessoais, porém a lei geral de proteção exige o consentimento do usuário
		para o uso de tais dados.
	\paragraph{}
		Para finalizar a palestra, Matheus Treuk nos lançou uma pergunta:
		\begin{center}		
			É estritamente necessário o utilização de dados biométricos em certas
			ocasiões?
		\end{center}
		A grande busca por dados hoje em dia faz com que a captação desse recurso
		fuja um pouco do controle, invadindo muitas vezes a privacidade dos
		indivíduos. Não podemos aceitar tais violações e invasões, para isso
		devemos não só buscar o conhecimento do que estão capturando e fazendo
		com nossas informações mas lutar por leis que tentem controlar e/ou 
		regularizar esse novo horizonte de troca de informações.		

\end{document}