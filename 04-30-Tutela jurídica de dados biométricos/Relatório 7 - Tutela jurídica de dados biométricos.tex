\documentclass[12pt]{article}

\usepackage[utf8]{inputenc}
\usepackage[portuguese]{babel}

\title{Relatório 7: Tutela jurídica de dados biométricos no Brasil}
\author{Mateus Agostinho dos Anjos\\NUSP: 9298191}
\date{\today}

\begin{document}
	\maketitle
	\paragraph{}
		Nesta palestra sobre Tutela jurídica de dados biométricos no Brasil
		o palestrante, Matheus Treuk, norteou nossa discussão começando com uma
		definição etimológica da palavra \textit{Biometria}, que significa, de maneira 
		simplificada, "medida humana".  Definido isso, categorizou biometria em
		\textit{fisiológicas} e \textit{fisionômicas}.
	\paragraph{}
		Dando sequência à introdução vimos que o conceito de biometria pode ser
		aplicado desde a pré história, pois os seres reconheciam uns aos outros pelas
		características faciais e continuando os relatos históricos vimos que na 
		mesopotâmia antiga já havia início do reconhecimento de digitais. Já a partir de
		1971 a geometria da mão passou a ser utilizada, em 1994 a íris humana e
		em 2010 o \textit{Facebook} começou com o reconhecimento facial no
		aplicativo de marcar pessoas, introduzindo uma questão sobre dados 
		pessoais e captação de informações biométricas sem que a maioria das
		pessoas saiba disso.
	\paragraph{}
		Após a introdução histórica, Matheus Treuk trouxe a discussão para o contexto
		brasileiro em que não existe uma definição de dados biométricos apresentando
		também alguns conceitos chaves sobre biometria. A partir deste momento
		tivemos um breve exemplo de como uma possível captura de dados biométricos
		é feita: a primeira medição serve como cadastro e salva os dados iniciais, após
		isso é feita uma segunda medição que servirá como comparação, para que
		sejam identificados os dados biométricos mais importantes e assim
		os dados são salvos.
	\paragraph{}
		Devemos lembrar que não basta fazer duas medições de quaisquer dados do
		corpo humano, para que um elemento seja um dado biométrico, portanto capaz
		de identificar um ser humano, ele deve possuir certas características, são
		elas:
		\begin{itemize}
			\item[•]
				Universal - Todas ou quase todas as pessoas possuem
			\item[•]
				Única - Devem ser apresentadas de forma única ou quase única em cada
				pessoa
			\item[•]
				Permanente - Permanente ou muito difícil de mudar		
		\end{itemize}			

		
\end{document}