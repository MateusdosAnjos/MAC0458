\documentclass[12pt]{article}

\usepackage[utf8]{inputenc}
\usepackage[portuguese]{babel}

\title{Relatório 9: Privacidade e proteção de dados pessoais}
\author{Mateus Agostinho dos Anjos\\NUSP 9298191}
\date{\today}

\begin{document}
	\maketitle
	\paragraph{}
		O palestrante Bruno Bioni iniciou a palestra diferenciando os conceitos
		de \textit{Direito à privacidade} e \textit{Proteção de dados pessoais}, pois
		muitas vezes confundimos o que eles realmente significam e isso dificulta
		tanto o entendimento quanto a discussão de novas medidas a serem tomadas
		em relação ao tema tratado.
	\paragraph{}
		Primeiro nos foi apresentada a \textit{Privacidade}, 	uma liberdade 
		negativa (confesso que não entendi muito bem o motivo
		de ser considerada negativa) que tem sua origem num passado distante
		uma vez que se baseia em tecnologias que bloqueiam o monitoramento das 
		pessoas em espaços privados, como por exemplo: cortinas, portas, janelas.
	\paragraph{}
		Portanto, nossas casas são ambientes privados, nosso quarto, etc. Sendo assim
		o direito à privacidade era considerado burguês, uma vez que apenas os mais 
		ricos possuíam espaços privados, enquanto os mais pobres moravam, por
		exemplo, em cortiços.
	\paragraph{}
		Já o 	\textit{Direito à proteção de dados pessoais} foi definido
		como uma liberdade positiva e garante que os dados pessoais coletados
		sejam usados somente para a finalidade a qual foram coletados. Caso
		dados pessoais coletados sejam usados para uma finalidade diferente
		da que foi informada aos indivíduos no momento da coleta temos a 
		violação do direito à proteção de dados pessoais.
	\paragraph{}
		Bruno Bioni exemplificou de várias formas como a proteção funciona,
		dentre os exemplos podemos citar:	
		\begin{itemize}
			\item			
				\textbf{Censo}: serve para política pública, coleta de dados para
    			determinada finalidade informada no momento da coleta que deve e
    			pode ser observada, garantindo controle sobre as informações 
    			que foram disponibilizadas. (Auto-determinação-informacional)
    		\item
    			\textbf{Software Tudo sobre Todos}: plataforma que utilizava dados 
    			publicamente acessíveis e agregava todas as informações:\\    
    			Sem violação de direito à privacidade, pois todos os dados eram públicos.
    			Violação de controle de dados, pois finalidade dos dados coletados não era 
    			a mesma de quando o indivíduo forneceu o dado, uma vez que a pessoa
    			que coletava os dados sobre outra pessoa poderia utilizar a informação
    			da maneira que desejasse.
    		\item
    			\textbf{Convenio TSF e Serasa}	: Serasa pede compartilhamento de 
				base de dados do TSF em troca de assinaturas digitais com a 
				finalidade de pegar possíveis fraudes usando dados de falecidos.	
			\item
				\textbf{Robô Rose do Serenata de Amor}: captava dados abertos de 
				senadores e deputados com a finalidade de fiscalizar certas medidas e
				gastos por parte destes indivíduos. Foi considerada uma atividade legal,
				pois o uso do dado estava de acordo com a finalidade para que foi
				disponibilizado.
		\end{itemize}
	\paragraph{}
		Depois dos exemplos, Bruno apresentou uma Lei discutida em 2018 que
		diz basicamente: \textit{dados pessoais publicamente disponibilizados ainda são 
		dados pessoais e devem ser submetidos a certo controle.}\\
		Para entendermos melhor o que essa lei significa devemos entender que
		dado pessoal é qualquer dado que nos identifica, ainda que de maneira remota,
		portanto endereço IP é, também, um dado pessoal.
	\paragraph{}
		Após pincelar as definições e estabelecer os conceitos básicos sobre o tema
		pudemos refletir um pouco sobre a importância de se controlar o uso
		dos dados, uma vez que práticas abusivas já foram relatadas, dentre elas
		tivemos o uso de informações como bateria do celular em aplicativos
		de caronas ou a precificação dinâmica em sites de voos.
	\paragraph{}
		Sendo assim, devemos regulamentar o uso dos dados pessoais, pois o	
		bom e velho consentimento dos "termos de uso" já se mostrou um instrumento 
		deficitário, uma vez que não garante a transparência de como os dados estão
        utilizados. É necessário a mudança de cultura da organização a partir da melhor
		comunicação entre computação e direito, para que os projetos sejam 
		desenvolvidos com a preocupação do uso de dados.
	\paragraph{}
		Seria interessante que os novos softwares fossem "privacy by design", ou seja,
		produto concebido a partir dos conceitos de privacidade e
		política de dados pessoais, mesmo que falho, pois as regras ainda
		não estão bem definidas.
	\paragraph{}
		Para isso devemos compreender a diferença entre privacidade e dados pessoais 
		a fim de proteger os direitos e estimular atividades econômicas que se baseiam 
		no uso destes dados e extrapolar essa bolha para outras áreas. Não devemos
		temer a regulamentação dos dados pessoais assim como tememos o código de 
		defesa do consumidor, pois são leis que ajudarão no progresso. Um bom
		exemplo do uso da regulamentação de forma positiva é a \textit{Brastemp} que 
		concebeu produtos e serviços mais seguros elevando-se no conceito dos 
		consumidores.
	\paragraph{}
		Portanto, devemos continuar a discutir privacidade e uso de dados, pois
		cada vez mais a coleta de informações se torna mais importante e sem
		a devida regulamentação o elo mais fraco, que somos nós, poderá sair
		extremamente prejudicado.	
\end{document}