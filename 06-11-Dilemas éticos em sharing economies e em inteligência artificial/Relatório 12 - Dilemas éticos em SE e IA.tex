Palestra 12: Dilemas éticos em sharing economies e em inteligência artificial

Introdução com contextualização do ambiente em que o tema foi iniciado com
regras de economia e alguns conceitos basicos.

revolução digital: facilita a transação entre os fatores economicos.

2016 - Barack Obama querendo ser empreendedor no Vale do Silicio - palestrante
diz que, hoje em dia, Obama não gostaria de empreender nesta área, pois
hoje em dia a crise de legitmidade da internet mudou muito de como as
coisas estão relacionadas. Relação com privacidade -> luta pela privacidade se 
tornou o epicentro da luta pelos direitos humanos.
A tecnologia da internet gerou mudanças gigantescas na organização da 
sociedade, mais do que outras tecnologias como o automovel, pois atinge
além de tudo a  subjetividade.

Abordagem crítica em relação a revolução digital - NY times, Le Monde, etc...

Tensão: Avanços tecnologicos trazem benefícios para a sociedade mas o
essencial dessas conquistas tecnologicas se apoiam em DADOS. As máquinas
"aprendem" com os dados que a alimentam. Portanto os dados podem ser 
considerados como bem comum da sociedade, porém tem servido para algumas
finalidades que não geram benefícios. (citou texto de Cedric Villani)

Menção de 4 Dimensões da perda de legitimidade:

    Vícios Digitais (Adam Alter): Estamos cada vez mais dependentes de 
    dispositivos digitais, não só relacionado aos aparelhos celulares e
    computadores, mas no tempo em que estamos conectados e mudança da relação
    entre pais e filhos.
    Skinner - Experiencia com o pombo, alavanca e comida aleatória.
    Ser humano é previsível, se achamos ter liberdade isso se deve a não
    conseguirmos identificar quais variáveis estão influenciando
    na decisão.
    Referência: Termo: "Captologia" - Pessoa:Fogg - Universidade:Stanford

    Concorrência: diferença de empresas dominantes a 20 anos atras e hoje
    em dia. Mercado é sistema de informações descentralizado em que voce
    só obtem informação do que é desejado após a compra do seu produto.
    Tecnologia permite uma antecipação do desejo (Banco de dados Amazon por
    exemplo). Inversão do processo de criação da publicidade. Tecnologia que
    deveria ajudar na manutenção dos preços acabou criando "dadopolios".
    Um dos fundadores de paypal: "Concorrencia é para fracos, vitoriosos 
    conseguem monopolio".
    Roosevelt - monopólio é ruim, dividiu a maior empresa de petróleo dos EUA

    Político: Tecnologia influenciando na política dos países, mídia
    social principalmente, ainda mais com Fake News. Ideia de Esfera
    pública fantasma, vetores de informação na qual não temos condições
    de discutir sobre o assunto.

    Uso dos dados para vigilância Policial: constante vigilância modificando
    o comportamento e dificultando nossa liberdade e anonimato

     