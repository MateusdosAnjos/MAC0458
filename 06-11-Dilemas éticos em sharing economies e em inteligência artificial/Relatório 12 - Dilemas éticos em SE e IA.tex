\documentclass[12pt]{article}

\usepackage[utf8]{inputenc}
\usepackage[portuguese]{babel}

\title{Relatório 12: Dilemas éticos em sharing economies e em inteligência artificial}
\author{Mateus Agostinho dos Anjos\\NUSP 9298191}
\date{\today}

\begin{document}
	\maketitle
	\paragraph{}
		O palestrante, Ricardo Abramovay, iniciou introduzindo e contextualizando
		o ambiente em que o tema foi iniciado, além de nos apresentar algumas
		regras de economia e conceitos básicos para a melhor compreensão
		do que viria a seguir.
	\paragraph{}
		A partir da introdução, nos foi mostrado os impactos da revolução digital,
		fator que facilitou a transação entre os fatores econômicos, modificando
		as formas de negócio e de como passamos a ver o mundo. Para exemplificar
		este ponto, Ricardo Abramovay citou Barack Obama que, em 2016, 
		mostrava um grande interesse no empreendedorismo no Vale do Silício,
		hoje em dia provavelmente Obama não gostaria de empreender nesta área, 
		pois a crise de legitimidade da internet mudou muito a forma de relação
		entre humano e tecnologia.
	\paragraph{}
		Introduzido o assunto sobre crise de legitimidade e mudança da relação
		com a tecnologia o palestrante definiu as diferentes relações e os
		motivos pelos quais essa crise vigora. Primeiro relacionou tecnologia com
		privacidade, tema também abordado em palestras anteriores, como
		o epicentro da luta pelos direitos humanos, uma vez que a relação entre
		os indivíduos acontece, cada vez mais, por intermédio da tecnologia e
		os direitos devem ser respeitados independentemente do meio pelo qual
		as pessoas utilizam para se relacionar.
	\paragraph{}
		O motivo pelo qual essa luta nasceu foi as mudanças gigantescas na 
		organização da sociedade geradas pela Internet, mais do que outras
		tecnologias como o automóvel, pois atinge além de tudo a subjetividade.
		Desta forma, regular as interações dentro desta nova tecnologia tem
		se mostrado um grande desafio, uma vez que toda essa subjetividade
		ainda não é compreendida em sua totalidade.
	\paragraph{}
		Para embasar seu ponto o palestrante citou algumas abordagens críticas
		em relação a revolução digital feitas pelo NY times, Le Monde e outros
		jornais famosos ao redor do mundo. Nestas abordagens foi identificada
		uma tensão dos avanços tecnológicos que trazem benefícios para a sociedade, 
		porém o essencial dessas conquistas tecnológicas se apoiam em 
		\textbf{dados}. As máquinas "aprendem" com os dados que a alimentam,
		portanto os dados podem ser considerados como bem comum da sociedade, 
		porém tem servido para algumas finalidades que não geram benefícios. (citou
		texto de Cedric Villani sobre este assunto).
	\paragraph{}
		Após abordar alguns motivos e tensões sobre Internet e novas tecnologias,
		Ricardo Abramovay iniciou a discussão sobre as quatro dimensões da perda
		de legitimidade, são elas \textbf{Vícios Digitais, Concorrência, Política} e 
		\textbf{Vigilância Policial}.
	\paragraph{}	
		Quando consideramos os \textbf{Vícios Digitais}, tema muito abordado
		por Adam Alter citado pelo palestrante, estamos cada vez mais 
		dependentes de dispositivos digitais, não só nos relacionando com aparelhos 
		celulares e computadores em si, mas também no tempo em que estamos 
		conectados. Tal fenômeno reflete, por exemplo, mudança da relação 
		entre pais e filhos.
    \paragraph{}
    	Ainda dentro deste tema de vícios digitais o palestrante nos apresentou
    	o experimento de Skinner, com o pombos alimentados utilizando uma 
    	alavanca que soltava comida aleatoriamente quando pressionada. A partir
    	disso o palestrante disse que o ser humano também é previsível, se achamos
    	ter liberdade isso se deve a não conseguirmos identificar quais variáveis 
    	estão influenciando nossa decisão. Tal conclusão se aplica ao vício digital,
    	pois certas empresas, por exemplo, utilizam os dados coletados para
    	identificar tais variáveis e incentivar, a partir da ilusão da nossa liberdade,
    	o consumo de seus produtos.
    	Abramovay deixou como referência neste assunto o termo \textit{Captologia}
    	criado por Fogg na  universidade de Stanford.
	\paragraph{}
		Passando para a segunda dimensão, \textbf{Concorrência}, o palestrante
	   	nos lembrou sobre a diferença das empresas dominantes a 20 anos atrás e 
	   	hoje em dia, sendo que na atualidade são as empresas de redes sociais,
	   	streaming, etc que dominam. 
	\paragraph{}	   	
	   	Tal fenômeno ocorre já que o mercado é um sistema de informações 
	   	descentralizado em que só obtém informação do que é desejado após a 
	   	compra do produto e com a evolução da tecnologia permitiu-se a 
	   	antecipação do desejo, por exemplo com o banco de dados da Amazon.
	   	Essa inversão do processo de criação da publicidade fez com que tais
	   	empresas conseguissem um maior número de vendas, sendo que agora
	   	a propaganda é feita a partir de seu desejo pelo produto e não para te 
	   	estimular a desejar o produto. (Você faz uma pesquisa sobre calçados
	   	uma vez e nas próximas semanas você recebe "magicamente" publicidades
	   	de calçados)
	\paragraph{}	   	
	   	Portanto, percebemos a tecnologia, que
    	deveria ajudar na manutenção dos preços, criando "dadopolios".
    	Citando um dos fundadores do paypal: "Concorrência é para fracos, 
    	vitoriosos conseguem monopólio", reforçando essa ideia de que tais
    	empresas lutam pelo domínio, não só econômico mas dos dados.
	\paragraph{}
		Passando para a terceira dimensão, temos a esfera \textbf{Política}
   		com a tecnologia influenciando na política dos países, principalmente
    	as mídias sociais, ainda mais com Fake News. O palestrante nos
    	apresentou a ideia de esfera pública fantasma, vetores de informação 
    	nos quais não temos condições de discutir sobre os assuntos, uma vez que
    	a informação chega porém não temos como respondê-la.
	\paragraph{}
		No final da palestra nos foi apresentado a última dimensão problemática
    	\textbf{Vigilância Policial}, como sendo a constante vigilância modificando
    	o comportamento e dificultando nossa liberdade e anonimato.
    \paragraph{}
    	O palestrante encerrou quando apresentou a última dimensão, uma vez que
    	nosso tempo chegava ao fim. Analisando os conceitos apresentados
    	percebemos que a tecnologia vem influenciando todas as nossas ações
    	e como o mundo se organiza, devemos aprender a lidar com tais mudanças
    	para que não sejamos manipulados e explorados por quem detém o poder,
    	poder este que hoje em dia vem em forma de dados.
\end{document}