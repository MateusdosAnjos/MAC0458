\documentclass[12pt]{article}

\usepackage[utf8]{inputenc}
\usepackage[portuguese]{babel}

\title{Relatório 12: Dilemas éticos em sharing economies e em inteligência artificial}
\author{Mateus Agostinho dos Anjos\\NUSP 9298191}
\date{\today}

\begin{document}
	\maketitle
	\paragraph{}
		O palestrante, Ricardo Abramovay, iniciou introduzindo e contextualizando
		o ambiente em que o tema foi iniciado, além de nos apresentar algumas
		regras de economia e conceitos básicos para a melhor compreensão
		do que viria a seguir.
	\paragraph{}
		A partir da introdução, nos foi mostrado os impactos da revolução digital,
		fator que facilitou a transação entre os fatores econômicos, modificando
		as formas de negócio e de como passamos a ver o mundo. Para exemplificar
		este ponto, Ricardo Abramovay citou Barack Obama que, em 2016, 
		mostrava um grande interesse no empreendedorismo no Vale do Silicio,
		hoje em dia provavelmente Obama não gostaria de empreender nesta área, 
		pois a crise de legitimidade da internet mudou muito a forma de relação
		entre humano e tecnologia.
	\paragraph{}
		Introduzido o assunto sobre crise de legitimidade e mudança da relação
		com a tecnologia o palestrante definiu as diferentes relações e os
		motivos pelos quais essa crise vigora. Primeiro relacionou tecnologia com
		privacidade, tema também abordado em palestras anteriores, como
		o epicentro da luta pelos direitos humanos, uma vez que a relação entre
		os indivíduos acontece, cada vez mais, por intermédio da tecnologia e
		os direitos devem ser respeitados independentemente do meio pelo qual
		as pessoas utilizam para se relacionar.
	\paragraph{}
		O motivo pelo qual essa luta nasceu foi as mudanças gigantescas na 
		organização da sociedade geradas pela Internet, mais do que outras
		tecnologias como o automóvel, pois atinge além de tudo a subjetividade.
		Desta forma, regular as interações dentro desta nova tecnologia tem
		se mostrado um grande desafio, uma vez que toda essa subjetividade
		ainda não é compreendida em sua totalidade.
	\paragraph{}
		Para embasar seu ponto o palestrante citou algumas abordagens críticas
		em relação a revolução digital feitas pelo NY times, Le Monde e outros
		jornais famosos ao redor do mundo. Nestas abordagens foi identificada
		uma tensão dos avanços tecnológicos que trazem benefícios para a sociedade, 
		porém o essencial dessas conquistas tecnológicas se apoiam em 
		\textbf{dados}. As máquinas "aprendem" com os dados que a alimentam,
		portanto os dados podem ser considerados como bem comum da sociedade, 
		porém tem servido para algumas finalidades que não geram benefícios. (citou
		texto de Cedric Villani sobre este assunto).
	\paragraph{}
		Após abordar alguns motivos e tensões sobre Internet e novas tecnologias,
		Ricardo Abramovay iniciou a discussão sobre as quatro dimensões da perda
		de legitimidade, são elas \textbf{Vícios Digitais, Concorrência, Política} e 
		\textbf{Vigilância Policial}.
	\paragraph{}	
		Quando consideramos os \textbf{Vícios Digitais}, tema muito abordado
		por Adam Alter citado pelo palestrante, estamos cada vez mais 
		dependentes de dispositivos digitais, não só nos relacionando com aparelhos 
		celulares e computadores em si, mas também no tempo em que estamos 
		conectados. Tal fenômeno reflete, por exemplo, mudança da relação 
		entre pais e filhos.
    
    
    Skinner - Experiencia com o pombo, alavanca e comida aleatória.
    Ser humano é previsível, se achamos ter liberdade isso se deve a não
    conseguirmos identificar quais variáveis estão influenciando
    na decisão.
    Referência: Termo: "Captologia" - Pessoa:Fogg - Universidade:Stanford

    Concorrência: diferença de empresas dominantes a 20 anos atras e hoje
    em dia. Mercado é sistema de informações descentralizado em que voce
    só obtem informação do que é desejado após a compra do seu produto.
    Tecnologia permite uma antecipação do desejo (Banco de dados Amazon por
    exemplo). Inversão do processo de criação da publicidade. Tecnologia que
    deveria ajudar na manutenção dos preços acabou criando "dadopolios".
    Um dos fundadores de paypal: "Concorrencia é para fracos, vitoriosos 
    conseguem monopolio".
    Roosevelt - monopólio é ruim, dividiu a maior empresa de petróleo dos EUA

    Político: Tecnologia influenciando na política dos países, mídia
    social principalmente, ainda mais com Fake News. Ideia de Esfera
    pública fantasma, vetores de informação na qual não temos condições
    de discutir sobre o assunto.

    Uso dos dados para vigilância Policial: constante vigilância modificando
    o comportamento e dificultando nossa liberdade e anonimato

     
\end{document}