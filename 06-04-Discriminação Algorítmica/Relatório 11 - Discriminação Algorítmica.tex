\documentclass[12pt]{article}

\usepackage[utf8]{inputenc}
\usepackage[portuguese]{babel}

\title{Relatório 11: Discriminação Algorítmica}
\author{Mateus Agostinho dos Anjos\\NUSP 9298191}
\date{\today}

\begin{document}
	\maketitle
	\paragraph{}
		A palestrante Marcela Mattiuzzo deu início à discussão de maneira 
		interativa, nos fazendo algumas perguntas e construindo as respostas
		conosco. Primeiro nos foi perguntado \textit{O que é um algoritmo?}
		e a resposta simplificada a qual chegamos foi: \textit{é um conjunto de
		operações ordenadas para a resolução de um determinado problema}.
		Nesta primeira resposta não ficamos presos aos algoritmos apenas
		computacionais, portanto uma receita de bolo também se enquadra no
		conceito de algoritmo.
	\paragraph{}
		Após essa discussão mais abrangente sobre o que é um algoritmo a
		palestrante tratou sobre os algoritmos computacionais, cuja diferença
		consiste em serem compostos por instruções ou operações bem definidas 
		para que a máquina seja capaz de interpretar.
	\paragraph{}
		Definida uma das palavras do tema passamos para o entendimento do
		termo \textit{discriminação}. Novamente de maneira iterativa chegamos
		à conclusão de que discriminação, de maneira simplificada, é a
		\textit{segregação de algo ou alguém considerando determinada 
		característica}. Do mesmo modo que foi feito anteriormente esta é
		uma descrição que não está limitada ao ambiente computacional.
	 \paragraph{}
	 	A fim de introduzir a 	\textit{discriminação algorítmica}, Marcela Mattiuzzo
	 	nos deu como exemplo análise de créditos. Esse processo pode levar em
	 	conta o local em que o indivíduo mora para calcular sua pontuação, porém
	 	esta é uma forma de discriminação, uma vez que o algoritmo que efetua
	 	o cálculo pode entender que moradores da periferia são "maus pagadores",
	 	sendo assim estaria segregando um conjunto de indivíduos levando em 
	 	conta o local de moradia (determinada característica). 
	\paragraph{}
		A palestrante atentou ao fato de que tais discriminações nem sempre são
		pensadas, ou seja, quem projeta o algoritmo muitas vezes nem pensou
		no fato de que esse problema pode acontecer ou nem tem conhecimento
		de que isso existe. A partir disso reforçou a importância da maior
		comunicação entre as áreas do direito e da computação, a fim de
		minimizar tais problemas. Para reforçar nosso conhecimento sobre
		o tema, a palestrante nos recomendou ler o livro de Cathy O'Neal
		chamado "Weapons of Math Destruction".
	\paragraph{}
		Explorando ainda mais a computação dentro do direito discutimos
		sobre como aplicar algoritmos para decisões jurídicas. Para isso
		é necessário definir de parâmetros para a criação de tais 
		algoritmos e definir certas generalizações para facilitar a
		tomada de decisões. A partir disso discutimos sobre as diferentes 
		formas de generalização.
	\paragraph{}
		Para discutir o assunto de generalizações a palestrante nos apresentou
		esse conceito de acordo com Fredrick Schauer. O autor diz que podemos 
		dividir as generalizações em estatisticamente corretas e estatisticamente
		não corretas.
	\paragraph{}
		Para os generalizações estatisticamente corretas temos:
		\begin{itemize}
			\item[•]		
				Universal, por exemplo "todos os seres humanos são mortais", sabemos
				que essa é uma generalização verdadeira para todos os seres humanos.
			\item[•]	
				Maioria, por exemplo "todo passaro voa", apesar de existirem exceções
				é de conhecimento geral que a maioria dos pássaros voam.
			\item[•]	
				Comparativa, por exemplo "todos os bulldogs têm problema no quadril",
				apesar de não ser verdade que todos, nem a maioria dos bulldogs têm
				problema no quadril é verdade que, em comparação com outros 
				cachorros,	os bulldogs apresentam mais frequentemente esse tipo
				de problema.
		\end{itemize}
	\paragraph{}
		Após a apresentação desses conceitos discutimos	quando podemos usar
		generalização por maioria ou comparativa, chegando à conclusão de que
		é necessário analisar o conteúdo da generalização e seu impacto no mundo 
		real antes de generalizar, uma vez que sem essa análise prévia podemos
		criar conceitos e até algoritmos discriminatórios que prejudiquem certa
		parte da população.
	\paragraph{}
		Vale lembrar que muitos dos modelos e algoritmos que utilizamos hoje em dia 
		buscam separar grupos a fim de encontrar padrões. Tal comportamento
		não é condenável quando o tratamento discriminatório não é usado
		para fins ilícitos ou abusivos, portanto nem todo algoritmo que
		segrega grupos e utiliza parâmetros para tais comparações é
		um algoritmo nocivo e muitos deles já se mostraram benéficos, além
		de precisos, para a sociedade em que vivemos.
	\paragraph{}	
		É importante, portanto, sabermos o que se enquadra em fins
		ilícitos ou abusivos, para evitarmos algoritmos que utilizem a
		generalização desta forma. É considerado como fim ilícito o uso
		de generalização que fere os escritos de alguma lei, ou seja, existe 
		uma lei escrita que pune tal ato. Já para fins abusivos não existe regra
		expressa impedindo o uso da generalização, porém o resultado fere 
		algum valor da sociedade.		
	\paragraph{}	
		Após discutirmos os modos de generalização e como eles influenciam
		na criação de algoritmos do ponto de vista jurídico a palestrante
		finalizou com algumas sugestões de como minimizar tais problemas,
		uma vez que é muito complexo e relativamente recente:\newline
		Devemos estar cientes de que tal problema existe (portanto a palestra
		foi o primeiro passo).\newline
		Juntarmos aspectos jurídicos com os aspectos técnicos  o que faria com que
		estas discussões fizessem parte do dia a dia de ambas as profissões,
		melhorando a criação de soluções.
	\paragraph{}
		Desta forma pudemos perceber que criar um algoritmo vai muito além
		da matemática e das analises envolvidas no projeto, vai muito além de
		fazer o algoritmo funcionar e devemos atentar ao fato de que vivemos
		em uma sociedade problemática e que a tecnologia deve auxiliar nas
		melhorias desta sociedade e não na administração dos preconceitos e
		discriminações que nela existem.	
\end{document}