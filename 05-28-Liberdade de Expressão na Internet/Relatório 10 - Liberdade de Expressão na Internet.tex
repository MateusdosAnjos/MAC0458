\documentclass[12pt]{article}

\usepackage[utf8]{inputenc}
\usepackage[portuguese]{babel}

\title{Relatório 10: Liberdade de Expressão na Internet}
\author{Mateus Agostinho dos Anjos\\NUSP 9298191}
\date{\today}

\begin{document}
	\maketitle
	\paragraph{}
		A palestrante Veridiana Alimonti começou apresentando algumas vantagens
		da liberdade de expressão na internet que facilitou o exercício de certos 
		direitos pelas pessoas.

Artigo 13. Convenção Americana: direito à liberdade de expressão deve ser
respeitado, não só em casos "favoráveis", mas também em discurso OFENSIVO,
INQUIETANTE E PERTURBADOR

exemplo com galileu - liberdade de expressão

declaração de direitos da mulher e da cidadão, na revolução francesa, 1793
guilhotinada

"Code is Speech" -> algoritmo de criptografia 
escrever código também é expressão humana, deve ser protegido por direito.

Discurso não protegido: Artigo 13, 5. A lei deve proibir propaganda a favor de
guerra, bem como toda apologia ao ódio.

liberdade na internet - Não deve haver censura prévia, mas com 
responsabilização posterior.

responsabilidade dos conteúdos não criadoos pelos intermediários?

Manila Principles on intermediary liability - 6 princípios (pesquisar)

Uma breve pincelada sobre o Marco civil e a responsabilidade das
plataformas.

Sobre censura PRIVADA:

Por conta das políticas da própria plataforma:
Facebook - Nudez feminina exemplo da índia(ministério público) - bomba napalm
(imagem da guerra do vietnam)


Censura às comunidades menores e marginalizadas, pois não concordam com o
discurso dominante = Censura online é reflexo das diferenças offline

Fake News: maneira fácil de escapar de críticas (Trump)

importante ter em vista: discursos sobre o governo são, em muitos casos,
exercício da democracia
Freedom on the Net

Brasil: Conjunto de leis com noções amplas sobre fake news e trata a
divulgação de fake news como delito penal.

Como lidar com notícias falsas? Transparência quanto a dados utilizados,
anúncios, motivo de receber os anúncios. Combater bots maliciosos que
divulgam notícias falsas. (Minar criptografia?)
\end{document}