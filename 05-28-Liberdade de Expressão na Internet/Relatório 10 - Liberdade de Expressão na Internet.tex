\documentclass[12pt]{article}

\usepackage[utf8]{inputenc}
\usepackage[portuguese]{babel}

\title{Relatório 10: Liberdade de Expressão na Internet}
\author{Mateus Agostinho dos Anjos\\NUSP 9298191}
\date{\today}

\begin{document}
	\maketitle
	\paragraph{}
		A palestrante Veridiana Alimonti começou apresentando algumas vantagens
		da liberdade de expressão na internet que facilitou o exercício de certos 
		direitos pelas pessoas, na sequência discutimos a Convenção Americana dos 
		direitos à liberdade de expressão quem devem ser respeitados, não só em casos
		"favoráveis", mas também em discursos ofensivos, inquietantes e perturbadores,
		a discussão se mostrou ainda mais interessante devido ao momento que vivemos
		em nossa sociedade.
	\paragraph{}
		Ressaltando a importância de se proteger a liberdade de expressão inclusive
		em discursos contrários ao pensamento majoritário, o que às vezes pode
		ser até ofensivo, a palestrante nos deu como exemplo o caso de Galileu Galilei.
		Na época a teoria de Galileu foi altamente desaprovada e mesmo estando certo
		foi condenado. Caso as leis de liberdade de expressão fossem uma preocupação
		nesta época teríamos resultados diferentes nos avanços da ciência, por exemplo.
	\paragraph{}
		Outro exemplo citado pela palestrante foi a	declaração de direitos da mulher e 
		da cidadã criado por Olympe de Gouges durante a revolução francesa o que 
		resultou em sua morte. Com isso podemos perceber que defender a liberdade
		de expressão vai além de proteger as pessoas, mas também serve para abrir
		espaço à discussão de novas ideias.
	\paragraph{}
		Depois da introdução histórica e de pincelar a importância de se manter
		a liberdade de expressão em lei, a palestrante deu exemplo em relação
		ao campo da computação com "Code is Speech". Citando um exemplo 
		polêmico de um algoritmo de criptografia open source em uma época que
		esses algoritmos eram considerados armas e produtos de exportação, a
		palestrante nos mostrou que escrever código é uma forma de expressão
		humana e deve ser protegida por direito, justificando o nome "Code is Speech".
	\paragraph{}
		Apesar da liberdade de expressão proteger certos discursos controversos
		existem certos teores de expressão que não são aceitos, como os definidos
		na lei de liberdade de pensamento e de expressão no artigo 13, 5: A lei deve 
		proibir propaganda a favor de guerra, bem como toda apologia ao ódio. O que
		tem se mostrado sensato.
	\paragraph{}
		Além disso, a internet deve ser regulamentada, porém não deve haver censura 
		prévia, mas com responsabilização posterior, como vimos em palestras como
		a do Marco Civil. As plataformas não são responsabilizadas imediatamente
		pelo conteúdo que transmitem, apenas quando são notificadas judicialmente.
		A partir disso Veridiana Alimonti nos indicou pesquisar sobre \textit{Manila 
		Principles on intermediary liability}.
	\paragraph{}
		Citando os princípios (em inglês para evitar perda de significado na tradução:
		\begin{itemize}
			\item[I]
				Intermediaries should be shielded by law from liability for third party 
				content.
			\item[II]
				Content must not be required to be restricted without an order by a 
				judicial authority.
			\item[III]
				Requests for restrictions of content must be clear, be unambiguous, and 
				follow due process.
			\item[IV]
				Laws and content restriction orders and practices must comply with the 
				tests of necessity and proportionality.
			\item[V]
				Laws and content restriction policies and practices must respect due 
				process.
			\item[VI]
				Transparency and accountability must be built into laws and content 
				restriction policies and practices.					
		\end{itemize}			
	\paragraph{}
		Após discutirmos as leis que garantem liberdade de expressão e que fazem
		certas censuras voltamos nosso olhar para as políticas privadas das empresas.
		Alguns exemplos foram:
		\textit{Facebook} e a nudez feminina em um retrato de índia, a imagem
		foi censurada por apresentar os seios da mulher, porém era uma
		imagem que retratava a realidade indígena e publicada pelo ministério público
		brasileiro. Mostrando que políticas privadas das empresas buscam defender
		o patrimônio e evitar processos, porém, por ser falho, causa problemas em
		certos casos. Outro exeplo mostrato foi a imagem da guerra do Vietnam de  
		pessoas dizimadas devido a bomba napalm.
	\paragraph{}		
		Apesar dos benefícios da liberdade de expressão, as censuras podem
		causar dano às comunidades menores e marginalizadas, pois não concordam com o
		discurso dominante e em sua luta podem ser consideradas demasiadas
		ofensivas, portanto a censura online é reflexo das diferenças offline, um
		fenômeno prejudicial, pois mantém a internet como um ambiente desigual.
	\paragraph{}
		Como o foco neste momento era a liberdade de expressão na internet a 
		palestrante aproveitou o momento para falar o papel das fake news como
		maneira fácil de escapar de críticas, citando o presidente Trump como exemplo.	
		Além disso ressaltou que discursos sobre o governo são, em muitos casos,
		exercício da democracia e que no Brasil existe um conjunto de leis com noções 
		amplas sobre fake news e que tratam a divulgação de fake news como delito 
		penal.
	\paragraph{}
		A palestra 
Como lidar com notícias falsas? Transparência quanto a dados utilizados,
anúncios, motivo de receber os anúncios. Combater bots maliciosos que
divulgam notícias falsas. (Minar criptografia?)

https://wilmap.law.stanford.edu/entries/manila-principles-intermediary-liability
\end{document}