\documentclass[12pt]{article}

\usepackage[utf8]{inputenc}
\usepackage[portuguese]{babel}

\title{Relatório 8: Direitos do consumidor e a internet}
\author{Mateus Agostinho dos Anjos\\NUSP 9298191}
\date{\today}

\begin{document}
	\maketitle
	\paragraph{}
		A palestrante Barbara Simões deu início a palestra apresentando alguns
		conceitos básicos sobre o código de defesa do consumidor e suas
		origens, citando Josephine Shaw Lowell e o surgimento do New York Consumers 
		League em 1890 como um dos primeiros meios de se defender o consumidor
		de práticas abusivas e danosas. Passando então para John F. Kennedy que
		criou 4 regras básicas do serviço, sendo elas:
		\begin{itemize}
			\item[•]
				Direito à segurança
			\item[•]	
                Direito à informação 
			\item[•]			
				Direito de escolha
			\item[•]				
				 Direito à reclamação
		\end{itemize}
	\paragraph{}
		A ONU então adicionou, em 1985, outras 4 regras:	
		\begin{itemize}
			\item[•]
				Direito a serviços essenciais
			\item[•]	
				Direito de acesso à justiça
			\item[•]			
                Direito à educação para o consumo
			\item[•]				
				 Direito a um ambiente saudável 
		\end{itemize}
	\paragraph{}
		A partir deste momento a defesa do consumidor se consolidava e o Código
		do Consumidor foi criado em 1990. Nesta época já estava muito claro
		o que era danoso ou não para o consumidor quando tratávamos de produtos
		físicos ou relacionados à exploração do trabalho,	porém o conceito de
		Internet ainda era recente e não sabíamos como lidar de fato com este 
		processo.
	\paragraph{}
		Para exemplificar a situação de defesa do consumidor na Internet, 
		Barbara Simões nos apresentou a evolução do Fair Credit Reporting Act
		de 1970 que funcionava como uma contagem de créditos do CEARA a partir
		de informações cadastrais de usuários. A fim de proteger o consumidor foi
		estipulado que apenas os dados dos 5 anos passados poderiam ser utilizados,
		além de outras mudanças que estão listadas no Código do Consumidor no
		artigo 43.
	\paragraph{}
		A partir disso pudemos perceber que já havia a preocupação de se regularizar
		a Internet e diminuir as potenciais práticas danosas aos usuários e 
		consumidores, entretanto os meios de se fazer isso ainda eram amplamente
		discutidos. Só em 2013 com o comércio eletrônico e em 2014 com o Marco
		Civil que as relações de consumo na Internet começaram a ser devidamente
		regulamentadas oferecendo maior proteção aos consumidores.
	\paragraph{}
		Mesmo com tais iniciativas alguns problemas permanecem 
		até hoje, como: \textit{Internet é telecomunicação?} Por enquanto a ANATEL
		não regulariza a Internet, pois queremos promover maior liberdade e
		criatividade na rede, porém as tecnologias de cabos marinhos utilizadas
		pelo \textit{Facebook} e pelo \textit{Google} são de telecomunicação,
		esse e outros exemplos fundamentam o questionamento.
	\paragraph{}
		Dentro dessas dificuldades levantadas questionamos alguns outros problemas
		que existem a respeito da defesa do consumidor dentro da Internet como:
		\begin{itemize}
			\item[•]
				Acesso universal à Internet - Nem todos possuem acesso à rede, isso
				mostra uma dificuldade estrutural enfrentada, pois as pessoas possuem
				o direito de acesso.
			\item[•]
				Convergência entre Internet e Telecomunicação - Como vamos 
				regulamentar essa convergência, como garantir que o acesso à
				Internet não seja feito como uma TV a cabo.
			\item[•]
				Neutralidade da rede - Como fazer com que a rede seja neutra, não
				dando prioridade para nenhum endereço web ou download, deixando
				o usuário livre para escolher o que melhor lhe convier.
			\item[•]
				Responsabilidade das Plataformas - Até que ponto uma plataforma
				deve se responsabilizar pelo conteúdo que ela transmite?				
		\end{itemize}
	\paragraph{}
		Além destes problemas citados uma nova gama está surgindo com a constante
		evolução de inteligência artificial e automatização de vários processos do
		cotidiano, constituindo novas dificuldades a serem tratadas no futuro. Sendo
		assim, algumas das dificuldades consistem em como garantir que o consumidor 
		não seja alvo de operações danosas ao utilizar os serviços ligados a essas 
		tecnologias e como fazer com que ele	tenha ferramentas para se defender 
		caso isso ocorra.
	\paragraph{}
		A palestrante encerrou com alguns questionamentos para refletirmos,
		me chamaram a atenção: \textit{A concorrência é um bom caminho?
		Regular mais é um bom caminho? Precisamos de novas leis?}
		Os processos judiciários no Brasil são extremamente lentos, criar novas
		leis talvez aumentasse a proteção no papel e a prática ficaria muito 
		defasada, entretanto apenas deixar que a concorrência estabilize este
		ambiente pode levar à grandes monopólios e à extinção de novas empresas.
		Pensando um pouco sobre tais problemas podemos fazer uma pequena ideia
		do que está acontecendo e da magnitude de tudo isso, o mais importante é que
		já está sendo pensado e novas propostas surgem nos encaminhando para
		a construção de uma rede menos nociva.				
\end{document}