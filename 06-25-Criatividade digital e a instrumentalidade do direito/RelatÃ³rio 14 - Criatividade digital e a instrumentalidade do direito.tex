\documentclass[12pt]{article}

\usepackage[utf8]{inputenc}
\usepackage[portuguese]{babel}

\title{Relatório 14: Criatividade Digital e a Instrumentalidade do Direito}
\author{Mateus Agostinho dos Anjos\\NUSP 9298191}
\date{\today}

\begin{document}
	\maketitle
	\paragraph{}
	

Noções básicas de como a base do direto funciona

O que a criatividade digital produz:
	criptomoeda
	disrupção no mercado financeira

Juridicização (termo importante)

Brasil -> Tradição romana -> Jurid. mal ou não juridic. num primeiro momento
EUA -> Tradição romana exterior -> diante de un fato novo o primeiro passo é
como juridc.

Por exemplo problema no Brasil, SEC XVIII, direito de patente

Inovações na Inglaterra muito devido a forma de se juridicizar 	

Propriedade Intelec + direito autoral = Lei de proteção de software
Brasil tentando administrar essa legislação

Mergulho teórico na Instrumentalidade do direito:
	Direito é só distinção, fenômeno da língua. Direito significa como a
	coletividade faz a distinção, baseando-se nos termos que a coletividade
	decidir.

	FIXAR: Direito é produto da língua.
	Língua é uma forma de organização de um povo.

Formas de legislar: de forma punitiva ou premiando
Faça SE NÃO Acontece algo diferente de Faça QUE você ganha algo

Legislação para controlar comportamentos: Se você quer que aconteca ou não quer
que não aconteça:
diferença entre omissão de socorro -> a ideia é que você socorra
matar uma pessoa -> a ideia é que voce NÂO MATE	

Diferentes elementos dos fatos ocorridos:

- Elemento espacial: Algo sempre acontece em algum lugar

- Elemento temporal: Algo acontece quando algo acontece

- Elemento materialidade: Operação de "compra e venda" (exemplo da mercadoria
na enchente)

- Elemento pessoalidade: deslocamento do imposto para outro (exemplo do seguro
de carro)

- Elemento quantitativo: Criar relação entre consequencia e causa (código de
hamurabi)

Caso de Mariano da Silva preso e vida acabada por "engano" gerou documentario

5 pilares são necessários para que se tome alguma providência.

A boa distinção faz com que haja proteção por leis.

Exemplo com Facebook: forma de adesão

Exemplo com propriedade das fotos e descontinuidade de uma plataforma:
Upload de foto passa a ser de propriedade do "face" para evitar o pagamento
de indenização caso descontinue a plataforma. Porém "face" passou a vender
essas informações, afinal eles detinham os direitos sob elas. PROBLEMA.
Inicio da discussão acerca do tema: Proteção de dados pessoais

Se pseudonimizar(dados que não possibilitam a identificação do indivíduo):
Empresa pode ter propriedade sob os dados para gerar inteligência

Importancia de se distinguir as coisas vem de quem vive no ambiente de uso do
caso que está querendo se distinguir

Em software: compra e venda é diferente de locação, estou vendendo o fonte do
software ou apenas permitindo o uso? (uso: Netflix)

Porém compra e venda e aluguel não da conta de todos os casos. Exemplo com
funk estralando no shopping iguatemi e exemplo de aluguel de fazenda.

Tais mudanças foram feitas a partir dos proprietários perceberem que essa
distinção deveria ser feita.
 
\end{document}