\documentclass[12pt]{article}

\usepackage[utf8]{inputenc}
\usepackage[portuguese]{babel}

\title{Relatório 14: Criatividade Digital e a Instrumentalidade do Direito}
\author{Mateus Agostinho dos Anjos\\NUSP 9298191}
\date{\today}

\begin{document}
	\maketitle
	\paragraph{}
		O palestrante Luís Soares, em sua palestra, nos explicou sobre a 
		criatividade digital e a instrumentalidade do direito, dando enfoque mais
		ao segundo tema. Para isso deu início introduzindo noções básicas de como
		a base do direto funciona, explicando sobre seus processos burocráticos.
	\paragraph{}
		Pincelando a criatividade digital o palestrante nos informou que a
		criatividade digital produz diversas coisas, dentre elas a criptomoeda
		e possíveis disrupções no mercado financeiro, tendo em vista que muitas
		das transações se dão pelo uso de novas tecnologias que surgem.
	\paragraph{}
		Voltando o enfoque para a instrumentalidade em si,  Luís Soares atentou
		a um termo muito importante nesta discussão: \textbf{Juridicização},
		como sendo a forma que a sociedade cria as leis sobre os diversos 
		comportamentos nela existentes.
	\paragraph{}
		Para exemplificar o novo termo introduzido foi feita uma comparação
		nos modos de juridicização brasileiros, cuja linha segue a tradição romana,
		e o modo de juridicização dos Estados Unidos que adotou uma forma
		posterior à tradicional romana. Vemos que no Brasil, ao ser notado um
		novo comportamento, não é criada nenhuma forma de regulamentá-lo ou
		é criada uma forma não ideal de regulamentação 	gerando problemas 
		futuros sobre a exploração de tal comportamento. Já nos EUA, ao
		se depararem com um novo comportamento, o primeiro passo é pensar
		em como juridicizá-lo.
	\paragraph{}
		A fim de embasar a importância da regulamentação o palestrante nos
		atentou ao fato do direito de patente ser um problema no Brasil
		até o século XVIII ao passo que a Inglaterra passou por grandiosos
		momentos de desenvolvimento devido, em grande parte, a sua 
		preocupação e maneira de se juridicizar as novas ideias que surgiam.
	\paragraph{}
		Trazendo a discussão para perto da realidade da ciência da computação
		tivemos como exemplo a lei de proteção de software que tenta
		unir elementos da propriedade intelectual com elementos dos
		direitos autorais a fim de regular este novo e crescente comportamento
		da sociedade que é a criação de softwares.
	\paragraph{}
		A partir deste momento a discussão tomou um rumo ainda mais teórico e
		técnico a respeito da instrumentalidade do direito. Nos foi falado que
		o direito nada mais é que a distinção entre elementos ou comportamentos,
		sendo exclusivamente um fenômeno da língua, ou seja, direito significa
		como a coletividade faz a distinção, baseando-se nos termos que ela 
		decidir.
	\paragraph{}
		Para reforçar essa definição devemos fixar que o \textbf{direito é
		produto da língua} e que a \textbf{língua é uma forma de organização de 
		um povo}.

Formas de legislar: de forma punitiva ou premiando
Faça SE NÃO Acontece algo diferente de Faça QUE você ganha algo

Legislação para controlar comportamentos: Se você quer que aconteca ou não quer
que não aconteça:
diferença entre omissão de socorro -> a ideia é que você socorra
matar uma pessoa -> a ideia é que voce NÂO MATE	

Diferentes elementos dos fatos ocorridos:

- Elemento espacial: Algo sempre acontece em algum lugar

- Elemento temporal: Algo acontece quando algo acontece

- Elemento materialidade: Operação de "compra e venda" (exemplo da mercadoria
na enchente)

- Elemento pessoalidade: deslocamento do imposto para outro (exemplo do seguro
de carro)

- Elemento quantitativo: Criar relação entre consequencia e causa (código de
hamurabi)

Caso de Mariano da Silva preso e vida acabada por "engano" gerou documentario

5 pilares são necessários para que se tome alguma providência.

A boa distinção faz com que haja proteção por leis.

Exemplo com Facebook: forma de adesão

Exemplo com propriedade das fotos e descontinuidade de uma plataforma:
Upload de foto passa a ser de propriedade do "face" para evitar o pagamento
de indenização caso descontinue a plataforma. Porém "face" passou a vender
essas informações, afinal eles detinham os direitos sob elas. PROBLEMA.
Inicio da discussão acerca do tema: Proteção de dados pessoais

Se pseudonimizar(dados que não possibilitam a identificação do indivíduo):
Empresa pode ter propriedade sob os dados para gerar inteligência

Importancia de se distinguir as coisas vem de quem vive no ambiente de uso do
caso que está querendo se distinguir

Em software: compra e venda é diferente de locação, estou vendendo o fonte do
software ou apenas permitindo o uso? (uso: Netflix)

Porém compra e venda e aluguel não da conta de todos os casos. Exemplo com
funk estralando no shopping iguatemi e exemplo de aluguel de fazenda.

Tais mudanças foram feitas a partir dos proprietários perceberem que essa
distinção deveria ser feita.
 
\end{document}