\documentclass[12pt]{article}

\usepackage[utf8]{inputenc}
\usepackage[portuguese]{babel}

\title{Relatório 14: Criatividade Digital e a Instrumentalidade do Direito}
\author{Mateus Agostinho dos Anjos\\NUSP 9298191}
\date{\today}

\begin{document}
	\maketitle
	\paragraph{}
		O palestrante Luís Soares, em sua palestra, nos explicou sobre a 
		criatividade digital e a instrumentalidade do direito, dando enfoque mais
		ao segundo tema. Para isso deu início introduzindo noções básicas de como
		a base do direto funciona, explicando sobre seus processos burocráticos.
	\paragraph{}
		Pincelando a criatividade digital o palestrante nos informou que a
		criatividade digital produz diversas coisas, dentre elas a criptomoeda
		e possíveis disrupções no mercado financeiro, tendo em vista que muitas
		das transações se dão pelo uso de novas tecnologias que surgem.
	\paragraph{}
		Voltando o enfoque para a instrumentalidade em si,  Luís Soares atentou
		a um termo muito importante nesta discussão: \textbf{Juridicização},
		como sendo a forma que a sociedade cria as leis sobre os diversos 
		comportamentos nela existentes.
	\paragraph{}
		Para exemplificar o novo termo introduzido foi feita uma comparação
		nos modos de juridicização brasileiros, cuja linha segue a tradição romana,
		e o modo de juridicização dos Estados Unidos que adotou uma forma
		posterior à tradicional romana. Vemos que no Brasil, ao ser notado um
		novo comportamento, não é criada nenhuma forma de regulamentá-lo ou
		é criada uma forma não ideal de regulamentação 	gerando problemas 
		futuros sobre a exploração de tal comportamento. Já nos EUA, ao
		se depararem com um novo comportamento, o primeiro passo é pensar
		em como juridicizá-lo.
	\paragraph{}
		A fim de embasar a importância da regulamentação o palestrante nos
		atentou ao fato do direito de patente ser um problema no Brasil
		até o século XVIII ao passo que a Inglaterra passou por grandiosos
		momentos de desenvolvimento devido, em grande parte, a sua 
		preocupação e maneira de se juridicizar as novas ideias que surgiam.
	\paragraph{}
		Trazendo a discussão para perto da realidade da ciência da computação
		tivemos como exemplo a lei de proteção de software que tenta
		unir elementos da propriedade intelectual com elementos dos
		direitos autorais a fim de regular este novo e crescente comportamento
		da sociedade que é a criação de softwares.
	\paragraph{}
		A partir deste momento a discussão tomou um rumo ainda mais teórico e
		técnico a respeito da instrumentalidade do direito. Nos foi falado que
		o direito nada mais é que a distinção entre elementos ou comportamentos,
		sendo exclusivamente um fenômeno da língua, ou seja, direito significa
		como a coletividade faz a distinção, baseando-se nos termos que ela 
		decidir.
	\paragraph{}
		Para reforçar essa definição devemos fixar que o \textbf{direito é
		produto da língua} e que a \textbf{língua é uma forma de organização de 
		um povo}.
	\paragraph{}
		Após apresentar o conceito importante sobre o direito, o palestrante
		discutiu sobre duas formas de legislar: punitiva ou premiada. A
		diferença pode ser entendida da seguinte maneira, para a forma
		punitiva temos: faça \textit{SE NÃO} acontece algo 	
		para a forma premiada temos: faça \textit{QUE você GANHA} algo.
	\paragraph{}
		Desta maneira é possível	legislar a fim de controlar comportamentos,
		caso queira que certo comportamento aconteça ou não aconteça.
		Percebemos isso com um exemplo simples entre a diferença de omissão de
		socorro cuja ideia é que você socorra a pessoa ferida e matar uma pessoa
		cuja ideia é que você não mate.
	\paragraph{}
		A partir destes conceitos o palestrante nos explicou sobre os	
		diferentes elementos dos fatos ocorridos, elementos que são
		levados em conta quando se julga algum comportamento pela
		lei ou quando se criará uma nova lei. São eles:
		\begin{itemize}
			\item[-]
				Elemento espacial: Algo sempre acontece em algum \textbf{lugar}
			\item[-]
				Elemento temporal: Algo acontece \textbf{quando} algo acontece
			\item[-]
				Elemento materialidade: Operação de "compra e venda", envolve
				um \textbf{produto em transação}
			\item[-]
				Elemento pessoalidade: deslocamento do imposto para terceiros,
				\textbf{quem é o responsável pelo pagamento?}
			\item[-]
				Elemento quantitativo: Criar relação entre \textbf{consequência e 
				causa} (exemplo simples: código de Hamurábi)
		\end{itemize}
	\paragraph{}
		Estes cinco pilares são necessários para que se tome alguma providência,
		portanto é imprescindível que uma boa distinção seja feita ao se
		juridicizar algo, pois a boa distinção faz com que haja proteção por leis.
		Tal distinção cabe a quem está inserido no ambiente de uso do
		caso que está querendo se distinguir, pois são essas pessoas que têm
		maior conhecimento sobre as diferenças relevantes.
	\paragraph{}
		Luís Soares encerrou a palestra citando alguns exemplos de como
		tais conceitos estão aplicados no nosso dia a dia, como no processo
		de adesão ao \textit{Facebook}, o upload de fotos no \textit{Orkut} e
		as possíveis consequências de se descontinuar uma plataforma que 
		possui fotos "privadas"  \ bem como a venda de tais informações (dados).
	\paragraph{}
		Fez menção sobre dados pseudonimizados (não possibilitam a identificação 
		do indivíduo) e seu uso em empresas a fim de gerar inteligência (machine
		learning). Finalizando com a distinção entre compra e venda de locação, 
		analisando o fato com um exemplo sobre a Netflix em que não compramos
		o software mas apenas o utilizamos.
	\paragraph{}
		Concluímos, portanto que entender as formas de se fazer o direito
		contribui para entendê-lo e evitarmos possíveis problemáticas quanto
		a eventos futuros. Além disso, é interessante sabermos que temos
		a possibilidade de interferir explicando quais são as diferenciações
		boas e necessárias a respeito do tema em que estamos inseridos,
		reforçando a ideia de que, para melhorarmos o sistema, é muito
		importante que haja o bom relacionamento entre as áreas,
		principalmente entre computação e direito.
\end{document}