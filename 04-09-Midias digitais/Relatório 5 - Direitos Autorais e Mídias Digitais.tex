\documentclass[10pt]{article}

\usepackage[utf8]{inputenc}
\usepackage[portuguese]{babel}
\usepackage{amsmath}

\title{Relatório 5: Direitos Autorais e Mídias Digitais}
\author{Mateus Agostinho dos Anjos\\NUSP: 9298191}
\date{09/04/2019}

\begin{document}
	\maketitle
	\paragraph{}
		A discussão dessa palestra teve base nos Direitos autorais sobre conteúdos 
		de Internet com enfoque para Streaming Musical e o mercado da música.
		A palestrante iniciou levantando uma questão: \textbf{Qual o motivo os artistas 
		têm para retirar conteúdo do \textit{Spotify?}} Aparentemente deveria ser 
		lucrativo, pois existe direitos autorais sobre esse conteúdo. A sequência da 
		palestra explicou, de modo simplificado, o funcionamento dessa repartição de 
		lucros e como são feitas as taxações sobre esses serviços.
	\paragraph{}
		Dando início à explicação refletimos sobre a pergunta \textbf{Pra que serve
		o direito autoral?} concluímos que, de forma simplificada e geral, o direito
		autoral serve para:
		\begin{center}				
			\begin{itemize}
				\item[ - ]
					Garantir controle sobre as cópias
				\item[ - ]
					Estimular produção de coisas novas
				\item[ - ]
					Remunerar os autores
				\item[ - ]
					Facilitar o acesso aos conteúdos pelo público, tirando o monopólio
					de preços das mãos do autor
				\item[ - ]
					Regularizar os conteúdos criados	
			\end{itemize}
		\end{center}
	\paragraph{}
		Após conhecermos um pouco mais sobre direito autoral a palestrante nos
		apresentou um pensamento recorrente na década de 90 de que a internet
		acabaria com o direito autoral, uma vez que a informação passava entre as
		pessoas de maneira cada vez mais rápida e seria impossível controlar a quem 
		pertencia cada conteúdo, iniciando assim uma baixa na indústria fonográfica.
	\paragraph{}
		Com os avanços da tecnologia desde a década de 90 e a constante busca de novos
		métodos para controlar direitos autorais na rede a indústria fonográfica voltou a 
		ficar em alta a partir de 2014 (dados de IFPI digital music report). Como exemplo
		nos foi apresentado o caso com o aplicativo de streaming \textit{Napster}, na
		época foi exigido um filtro para verificação de direitos autorais nos uploads dos
		arquivos, porém sem condições de projetar tal filtro o \textit{Napster} foi
		abolido e retornaria modernizado somente algum tempo depois, exemplificando
		a adequação das tecnologias para a taxação dos direitos autorais.
	\paragraph{}
		A partir deste momento novas plataformas de streaming começaram a surgir
		e como consequência houveram mudanças no acesso e no mercado de tais 
		mídias, levando isso em consideração a palestrante trouxe a realidade
		brasileira para discussão: \textbf{Qual o nível de acesso a esse tipo de mídia
		no Brasil?} sabemos que o acesso à internet ainda é limitado em nosso país
		além disso o acesso à esses serviços muitas vezes é feito utilizando cartão
		de crédito e, segundo dados de 2013, apenas 52\% dos brasileiros possuíam
		cartão de crédito.
	\paragraph{}
			Como forma de contornar problemas a pirataria entrou em debate e
			questionamos os dados que são apresentados sobre pirataria, pois
			muitas vezes são dados obscuros que as próprias empresas produtoras
			de mídias divulgam. Claro que não vimos pirataria como um bom negócio
			e não nos aprofundamos na discussão deste tema, apenas fomos alertados
			para os dados que talvez sejam manipulados.
	\paragraph{}
		Outro conceito que surgiu junto com o avanço da internet é que ela seria
		capaz de abolir a necessidade de um intermediário entre o público e o
		autor. Esse fica facilitado em alguns casos como visto no \textit{YouTube},
		porém aplicativos como \textit{Spotify} ainda são intermediário dessa relação 
		entre autor e público.
	\paragraph{}
		Saindo um pouco da abordagem histórica e social, a palestrante se aprofundou mais na
		explicação	do funcionamento da cobrança e divisão sobre direitos autorais	
		revelando dois tipos:
		\begin{itemize}
			\item[1.]					
				Direito do Autor
					\subitem
						Autor da melodia
					\subitem	
						Autor da letra		
			\item[2.]
				Direitos Conexos
				\subitem
					Intérprete
				\subitem	
					Músico Executante
				\subitem	
					Produtor Fonográfico
		\end{itemize}
		Em seguida dividiu direitos autorais em dois:
		\begin{itemize}
			\item[1.]					
				Direitos de Reprodução
					\subitem
						Basicamente as cópias de Cd's, partituras, etc.		
			\item[2.]
				Direitos de Execução Pública
				\subitem
					Execuções em shows, bares, clubes, etc.
		\end{itemize}	
	\paragraph{}
		Devido ao fato de ser muito difícil para os artistas arrecadarem
		o dinheiro pelos direitos autorais corretos e de fiscalizarem a execução 
		de suas obras, foi criado o ECAD, um órgão responsável por arrecadar
		e distribuir a renda proveniente de direitos autorais, como forma de
		regularizar e organizar o processo.	
	\paragraph{}
		Os problemas no ECAD com streaming não tardaram a aparecer e devem - se ao fato, principalmente, da dificuldade
		em decidir qual tipo de direito autoral será cobrado e qual o contrato
		que será feito: Streaming é direito de reprodução ou direito de execução
		pública? O contrato pelos aplicativos deve ser feito com o ECAD ou com as
		empresas produtoras das mídias? Cada um defende seu ponto tentando
		facilitar os contratos ou arrecadar mais dinheiro.
	\paragraph{}
		Como forma de amenizar os problemas do ECAD os órgãos ABER e ABEM se
		fundiram para formar a UBEM numa tentativa de representar tanto os 
		autores quanto as editoras. Desta forma o caminho da remuneração se dá
		desta maneira:\\
		Spotify $->_\%$ UBEM $->_\%$ Editoras $->_\%$ Autores\\
		ou seja, os autores recebem uma pequenina porcentagem do que é arrecadado
		pelo Spotify, que ainda deve pagar às grandes gravadoras e aos artistas\\
		Spotify $->_\%$ Grandes Gravadoras $->_\%$ Artistas\\
		portanto o dinheiro proveniente de direitos autorais, para autores e artistas, é
		muito reduzido, devido às fatias recolhidas pelos intermediários, desta forma
		muitos preferem outros métodos para ganhar dinheiro, tentando eliminar
		os intermediários.
	\paragraph{}
		Nesta apresentação geral pudemos perceber que os sistemas de cobrança e
		as formas de produção evoluem junto com a tecnologia, órgãos de fiscalização
		buscam se modernizar a fim de suprir necessidades das tecnologias atuais
		e autores e produtores buscam novos métodos de utilizarem as novas
		tecnologias de forma rentável, sendo assim as reformas no modo de
		abordar certas situações devem ser feitas constantemente, mesmo que
		não sejam tão triviais, a fim de manter uma ordem mínima na produção dos
		conteúdos e na arrecadação dos lucros.

\end{document}