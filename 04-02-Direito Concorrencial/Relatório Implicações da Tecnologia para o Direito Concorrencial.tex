\documentclass{article}

\usepackage[utf8]{inputenc}
\usepackage[portuguese]{babel}

\title{Relatório: Implicações da Tecnologia para o Direito Concorrencial}
\author{Mateus Agostinho dos Anjos\\
NUSP: 9298191}
\date{08/04/2019}

\begin{document}
	\maketitle
	\paragraph{}
		A palestra teve início com uma introdução sobre direito concorrencial,
		nos apresentando a noção de vários agentes num único mercado competitivo
		que competem por: preços, clientes, lucros, etc. 
		Desta forma, a livre concorrência norteia a criação e manutenção de outras
		leis criadas para manter, de forma menos predatória, a livre concorrência.
	\paragraph{}
		Após explicado os conceitos iniciais a palestrante discutiu um pouco sobre
		as políticas de defesa da concorrência apresentando o órgão responsável
		por gerir essas normas, o CADE, e ramificando a política de defesa da 
		concorrência em três ramos a fim de explicar de maneira mais clara as
		funções dessa política.
		\begin{itemize}
			\item
				Promover Concorrência
				\subitem
					Advocacia da concorrência é uma área que trata sobre os
					problemas que a concorrência pode gerar, envolvendo, entre
					outras coisas, a relação entre defesa da concorrência e defesa
					do consumidor, defesa da concorrência em licitação, aspectos 
					práticos de licitações e conceitos básicos de análise econômica do 
					direito.
			\item
				Controle Preventivo
				\subitem
					Controle de estruturas: CADE precisa aprovar operações entre
					grandes empresas de forma a manter a livre concorrência,
					evitando monopólios. Ex: Fusão entre SADIA e Perdigão.
			\item
				Controle Repressivo
				\subitem
					Neste ramo temos o CADE mais ativo no processo, tendo como
					função PUNIR as empresas que prejudicarem a livre concorrência.
					\\
					Ex: Preços predatórios.		 
		\end{itemize}
		Na sequência tivemos uma explicação sobre os tipos de condutas que
		prejudicam a livre concorrência:
		\begin{itemize}				
			\item
				Cooperativa
				\subitem
					Ex: Cartel
			\item
				Unilateral
				\subitem
					Empresa muito poderosa capaz de manipular o mercado utilizando
					preços ou comprando concorrência.
		\end{itemize}
	\paragraph{}
		Passada essa introdução de alguns termos relacionados ao direito entramos em
		um discussão sobre o papel da tecnologia nesse contexto, entendendo como
		"Novas tecnologias" os termos: Big Data, Algoritmos, Plataformas e Inteligência
		Artificial. 
	\paragraph{}	
		Uma das dificuldades apresentadas relacionada à tecnologia foi:
		Entender o mercado das mídias digitais a fim de avaliar o impacto de fusões de 
		empresas. Algumas questões foram levantadas sobre esse assunto:
		\begin{itemize}
			\item[]
				Como avaliaremos a competição entre $youtube$ e $facebook$? 
			\item[]	
				Qual o impacto de uma fusão entre essas duas empresas? 
			\item[]
				Será que isso acabaria com a livre concorrência, pois criaria uma super
				empresa muito poderosa?
		\end{itemize}
	\paragraph{}	
		A formação de empresas muito poderosas dificulta a livre concorrência, pois antes
		de nascer um possível concorrente estas empresas compram suas ideias e 
		ferramentas, ou seja, englobam o concorrente antes que ele vire uma 
		concorrência, aumentando seu poder e domínio sobre o mercado. Avaliar e conter
		tais atos prejudiciais à livre concorrência é simples quando o produto ou serviço
		é físico, pois a avaliação sobre mercado é conhecida e mais simples.
		A grande questão que se mantém aberta é: \textbf{Como avaliar os serviços 
		digitais?}
	\paragraph{}
		Encaminhando para o final da palestra tivemos uma discussão sobre o quão
		gratuito são esses serviços digitais. Grandes empresas que trabalham com
		mídias digitais, como redes sociais, obtém muitos dados sobre nossas vidas	
		e utilizam tais informações para gerar conteúdos mais atrativos, controlar
		opiniões, manipular massas e influenciar no consumo de outros produtos, 
		portanto o poder gerado pelos dados hoje em dia está em ascensão vertiginosa,
		em grande parte devido à falta de conhecimento, dos usuários dessas redes, de 
		que estão coletando dados pessoais e produzindo a partir disso.
	\paragraph{}
		Tal domínio sobre os dados pode gerar uma falsa sensação de escolha nos
		consumidores prejudicando muito a livre concorrência, além de fortalecer
		grandes empresas criando impérios digitais. Portanto, cabe aos órgãos como
		o CADE encontrar uma maneira de controlar esse domínio e manter a livre
		concorrência. Para isso é necessário uma reforma em tais órgãos a fim de
		ampliar o conhecimento sobre as novas tecnologias, uma vez que é quase 
		impossível controlar o que não conhece ou não entende.
\end{document}